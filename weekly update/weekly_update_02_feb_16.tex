


% Header, overrides base

    % Make sure that the sphinx doc style knows who it inherits from.
    \def\sphinxdocclass{article}

    % Declare the document class
    \documentclass[letterpaper,10pt,english]{/anaconda/lib/python2.7/site-packages/sphinx/texinputs/sphinxhowto}

    % Imports
    \usepackage[utf8]{inputenc}
    \DeclareUnicodeCharacter{00A0}{\\nobreakspace}
    \usepackage[T1]{fontenc}
    \usepackage{babel}
    \usepackage{times}
    \usepackage{import}
    \usepackage[Bjarne]{/anaconda/lib/python2.7/site-packages/sphinx/texinputs/fncychap}
    \usepackage{longtable}
    \usepackage{/anaconda/lib/python2.7/site-packages/sphinx/texinputs/sphinx}
    \usepackage{multirow}

    \usepackage{amsmath}
    \usepackage{amssymb}
    \usepackage{ucs}
    \usepackage{enumerate}

    % Used to make the Input/Output rules follow around the contents.
    \usepackage{needspace}

    % Pygments requirements
    \usepackage{fancyvrb}
    \usepackage{color}
    % ansi colors additions
    \definecolor{darkgreen}{rgb}{.12,.54,.11}
    \definecolor{lightgray}{gray}{.95}
    \definecolor{brown}{rgb}{0.54,0.27,0.07}
    \definecolor{purple}{rgb}{0.5,0.0,0.5}
    \definecolor{darkgray}{gray}{0.25}
    \definecolor{lightred}{rgb}{1.0,0.39,0.28}
    \definecolor{lightgreen}{rgb}{0.48,0.99,0.0}
    \definecolor{lightblue}{rgb}{0.53,0.81,0.92}
    \definecolor{lightpurple}{rgb}{0.87,0.63,0.87}
    \definecolor{lightcyan}{rgb}{0.5,1.0,0.83}

    % Needed to box output/input
    \usepackage{tikz}
        \usetikzlibrary{calc,arrows,shadows}
    \usepackage[framemethod=tikz]{mdframed}

    \usepackage{alltt}

    % Used to load and display graphics
    \usepackage{graphicx}
    \graphicspath{ {figs/} }
    \usepackage[Export]{adjustbox} % To resize

    % used so that images for notebooks which have spaces in the name can still be included
    \usepackage{grffile}


    % For formatting output while also word wrapping.
    \usepackage{listings}
    \lstset{breaklines=true}
    \lstset{basicstyle=\small\ttfamily}
    \def\smaller{\fontsize{9.5pt}{9.5pt}\selectfont}

    %Pygments definitions
    
\makeatletter
\def\PY@reset{\let\PY@it=\relax \let\PY@bf=\relax%
    \let\PY@ul=\relax \let\PY@tc=\relax%
    \let\PY@bc=\relax \let\PY@ff=\relax}
\def\PY@tok#1{\csname PY@tok@#1\endcsname}
\def\PY@toks#1+{\ifx\relax#1\empty\else%
    \PY@tok{#1}\expandafter\PY@toks\fi}
\def\PY@do#1{\PY@bc{\PY@tc{\PY@ul{%
    \PY@it{\PY@bf{\PY@ff{#1}}}}}}}
\def\PY#1#2{\PY@reset\PY@toks#1+\relax+\PY@do{#2}}

\expandafter\def\csname PY@tok@gd\endcsname{\def\PY@tc##1{\textcolor[rgb]{0.63,0.00,0.00}{##1}}}
\expandafter\def\csname PY@tok@gu\endcsname{\let\PY@bf=\textbf\def\PY@tc##1{\textcolor[rgb]{0.50,0.00,0.50}{##1}}}
\expandafter\def\csname PY@tok@gt\endcsname{\def\PY@tc##1{\textcolor[rgb]{0.00,0.27,0.87}{##1}}}
\expandafter\def\csname PY@tok@gs\endcsname{\let\PY@bf=\textbf}
\expandafter\def\csname PY@tok@gr\endcsname{\def\PY@tc##1{\textcolor[rgb]{1.00,0.00,0.00}{##1}}}
\expandafter\def\csname PY@tok@cm\endcsname{\let\PY@it=\textit\def\PY@tc##1{\textcolor[rgb]{0.25,0.50,0.50}{##1}}}
\expandafter\def\csname PY@tok@vg\endcsname{\def\PY@tc##1{\textcolor[rgb]{0.10,0.09,0.49}{##1}}}
\expandafter\def\csname PY@tok@m\endcsname{\def\PY@tc##1{\textcolor[rgb]{0.40,0.40,0.40}{##1}}}
\expandafter\def\csname PY@tok@mh\endcsname{\def\PY@tc##1{\textcolor[rgb]{0.40,0.40,0.40}{##1}}}
\expandafter\def\csname PY@tok@go\endcsname{\def\PY@tc##1{\textcolor[rgb]{0.53,0.53,0.53}{##1}}}
\expandafter\def\csname PY@tok@ge\endcsname{\let\PY@it=\textit}
\expandafter\def\csname PY@tok@vc\endcsname{\def\PY@tc##1{\textcolor[rgb]{0.10,0.09,0.49}{##1}}}
\expandafter\def\csname PY@tok@il\endcsname{\def\PY@tc##1{\textcolor[rgb]{0.40,0.40,0.40}{##1}}}
\expandafter\def\csname PY@tok@cs\endcsname{\let\PY@it=\textit\def\PY@tc##1{\textcolor[rgb]{0.25,0.50,0.50}{##1}}}
\expandafter\def\csname PY@tok@cp\endcsname{\def\PY@tc##1{\textcolor[rgb]{0.74,0.48,0.00}{##1}}}
\expandafter\def\csname PY@tok@gi\endcsname{\def\PY@tc##1{\textcolor[rgb]{0.00,0.63,0.00}{##1}}}
\expandafter\def\csname PY@tok@gh\endcsname{\let\PY@bf=\textbf\def\PY@tc##1{\textcolor[rgb]{0.00,0.00,0.50}{##1}}}
\expandafter\def\csname PY@tok@ni\endcsname{\let\PY@bf=\textbf\def\PY@tc##1{\textcolor[rgb]{0.60,0.60,0.60}{##1}}}
\expandafter\def\csname PY@tok@nl\endcsname{\def\PY@tc##1{\textcolor[rgb]{0.63,0.63,0.00}{##1}}}
\expandafter\def\csname PY@tok@nn\endcsname{\let\PY@bf=\textbf\def\PY@tc##1{\textcolor[rgb]{0.00,0.00,1.00}{##1}}}
\expandafter\def\csname PY@tok@no\endcsname{\def\PY@tc##1{\textcolor[rgb]{0.53,0.00,0.00}{##1}}}
\expandafter\def\csname PY@tok@na\endcsname{\def\PY@tc##1{\textcolor[rgb]{0.49,0.56,0.16}{##1}}}
\expandafter\def\csname PY@tok@nb\endcsname{\def\PY@tc##1{\textcolor[rgb]{0.00,0.50,0.00}{##1}}}
\expandafter\def\csname PY@tok@nc\endcsname{\let\PY@bf=\textbf\def\PY@tc##1{\textcolor[rgb]{0.00,0.00,1.00}{##1}}}
\expandafter\def\csname PY@tok@nd\endcsname{\def\PY@tc##1{\textcolor[rgb]{0.67,0.13,1.00}{##1}}}
\expandafter\def\csname PY@tok@ne\endcsname{\let\PY@bf=\textbf\def\PY@tc##1{\textcolor[rgb]{0.82,0.25,0.23}{##1}}}
\expandafter\def\csname PY@tok@nf\endcsname{\def\PY@tc##1{\textcolor[rgb]{0.00,0.00,1.00}{##1}}}
\expandafter\def\csname PY@tok@si\endcsname{\let\PY@bf=\textbf\def\PY@tc##1{\textcolor[rgb]{0.73,0.40,0.53}{##1}}}
\expandafter\def\csname PY@tok@s2\endcsname{\def\PY@tc##1{\textcolor[rgb]{0.73,0.13,0.13}{##1}}}
\expandafter\def\csname PY@tok@vi\endcsname{\def\PY@tc##1{\textcolor[rgb]{0.10,0.09,0.49}{##1}}}
\expandafter\def\csname PY@tok@nt\endcsname{\let\PY@bf=\textbf\def\PY@tc##1{\textcolor[rgb]{0.00,0.50,0.00}{##1}}}
\expandafter\def\csname PY@tok@nv\endcsname{\def\PY@tc##1{\textcolor[rgb]{0.10,0.09,0.49}{##1}}}
\expandafter\def\csname PY@tok@s1\endcsname{\def\PY@tc##1{\textcolor[rgb]{0.73,0.13,0.13}{##1}}}
\expandafter\def\csname PY@tok@sh\endcsname{\def\PY@tc##1{\textcolor[rgb]{0.73,0.13,0.13}{##1}}}
\expandafter\def\csname PY@tok@sc\endcsname{\def\PY@tc##1{\textcolor[rgb]{0.73,0.13,0.13}{##1}}}
\expandafter\def\csname PY@tok@sx\endcsname{\def\PY@tc##1{\textcolor[rgb]{0.00,0.50,0.00}{##1}}}
\expandafter\def\csname PY@tok@bp\endcsname{\def\PY@tc##1{\textcolor[rgb]{0.00,0.50,0.00}{##1}}}
\expandafter\def\csname PY@tok@c1\endcsname{\let\PY@it=\textit\def\PY@tc##1{\textcolor[rgb]{0.25,0.50,0.50}{##1}}}
\expandafter\def\csname PY@tok@kc\endcsname{\let\PY@bf=\textbf\def\PY@tc##1{\textcolor[rgb]{0.00,0.50,0.00}{##1}}}
\expandafter\def\csname PY@tok@c\endcsname{\let\PY@it=\textit\def\PY@tc##1{\textcolor[rgb]{0.25,0.50,0.50}{##1}}}
\expandafter\def\csname PY@tok@mf\endcsname{\def\PY@tc##1{\textcolor[rgb]{0.40,0.40,0.40}{##1}}}
\expandafter\def\csname PY@tok@err\endcsname{\def\PY@bc##1{\setlength{\fboxsep}{0pt}\fcolorbox[rgb]{1.00,0.00,0.00}{1,1,1}{\strut ##1}}}
\expandafter\def\csname PY@tok@kd\endcsname{\let\PY@bf=\textbf\def\PY@tc##1{\textcolor[rgb]{0.00,0.50,0.00}{##1}}}
\expandafter\def\csname PY@tok@ss\endcsname{\def\PY@tc##1{\textcolor[rgb]{0.10,0.09,0.49}{##1}}}
\expandafter\def\csname PY@tok@sr\endcsname{\def\PY@tc##1{\textcolor[rgb]{0.73,0.40,0.53}{##1}}}
\expandafter\def\csname PY@tok@mo\endcsname{\def\PY@tc##1{\textcolor[rgb]{0.40,0.40,0.40}{##1}}}
\expandafter\def\csname PY@tok@kn\endcsname{\let\PY@bf=\textbf\def\PY@tc##1{\textcolor[rgb]{0.00,0.50,0.00}{##1}}}
\expandafter\def\csname PY@tok@mi\endcsname{\def\PY@tc##1{\textcolor[rgb]{0.40,0.40,0.40}{##1}}}
\expandafter\def\csname PY@tok@gp\endcsname{\let\PY@bf=\textbf\def\PY@tc##1{\textcolor[rgb]{0.00,0.00,0.50}{##1}}}
\expandafter\def\csname PY@tok@o\endcsname{\def\PY@tc##1{\textcolor[rgb]{0.40,0.40,0.40}{##1}}}
\expandafter\def\csname PY@tok@kr\endcsname{\let\PY@bf=\textbf\def\PY@tc##1{\textcolor[rgb]{0.00,0.50,0.00}{##1}}}
\expandafter\def\csname PY@tok@s\endcsname{\def\PY@tc##1{\textcolor[rgb]{0.73,0.13,0.13}{##1}}}
\expandafter\def\csname PY@tok@kp\endcsname{\def\PY@tc##1{\textcolor[rgb]{0.00,0.50,0.00}{##1}}}
\expandafter\def\csname PY@tok@w\endcsname{\def\PY@tc##1{\textcolor[rgb]{0.73,0.73,0.73}{##1}}}
\expandafter\def\csname PY@tok@kt\endcsname{\def\PY@tc##1{\textcolor[rgb]{0.69,0.00,0.25}{##1}}}
\expandafter\def\csname PY@tok@ow\endcsname{\let\PY@bf=\textbf\def\PY@tc##1{\textcolor[rgb]{0.67,0.13,1.00}{##1}}}
\expandafter\def\csname PY@tok@sb\endcsname{\def\PY@tc##1{\textcolor[rgb]{0.73,0.13,0.13}{##1}}}
\expandafter\def\csname PY@tok@k\endcsname{\let\PY@bf=\textbf\def\PY@tc##1{\textcolor[rgb]{0.00,0.50,0.00}{##1}}}
\expandafter\def\csname PY@tok@se\endcsname{\let\PY@bf=\textbf\def\PY@tc##1{\textcolor[rgb]{0.73,0.40,0.13}{##1}}}
\expandafter\def\csname PY@tok@sd\endcsname{\let\PY@it=\textit\def\PY@tc##1{\textcolor[rgb]{0.73,0.13,0.13}{##1}}}

\def\PYZbs{\char`\\}
\def\PYZus{\char`\_}
\def\PYZob{\char`\{}
\def\PYZcb{\char`\}}
\def\PYZca{\char`\^}
\def\PYZam{\char`\&}
\def\PYZlt{\char`\<}
\def\PYZgt{\char`\>}
\def\PYZsh{\char`\#}
\def\PYZpc{\char`\%}
\def\PYZdl{\char`\$}
\def\PYZhy{\char`\-}
\def\PYZsq{\char`\'}
\def\PYZdq{\char`\"}
\def\PYZti{\char`\~}
% for compatibility with earlier versions
\def\PYZat{@}
\def\PYZlb{[}
\def\PYZrb{]}
\makeatother


    %Set pygments styles if needed...
    
        \definecolor{nbframe-border}{rgb}{0.867,0.867,0.867}
        \definecolor{nbframe-bg}{rgb}{0.969,0.969,0.969}
        \definecolor{nbframe-in-prompt}{rgb}{0.0,0.0,0.502}
        \definecolor{nbframe-out-prompt}{rgb}{0.545,0.0,0.0}

        \newenvironment{ColorVerbatim}
        {\begin{mdframed}[%
            roundcorner=1.0pt, %
            backgroundcolor=nbframe-bg, %
            userdefinedwidth=1\linewidth, %
            leftmargin=0.1\linewidth, %
            innerleftmargin=0pt, %
            innerrightmargin=0pt, %
            linecolor=nbframe-border, %
            linewidth=1pt, %
            usetwoside=false, %
            everyline=true, %
            innerlinewidth=3pt, %
            innerlinecolor=nbframe-bg, %
            middlelinewidth=1pt, %
            middlelinecolor=nbframe-bg, %
            outerlinewidth=0.5pt, %
            outerlinecolor=nbframe-border, %
            needspace=0pt
        ]}
        {\end{mdframed}}
        
        \newenvironment{InvisibleVerbatim}
        {\begin{mdframed}[leftmargin=0.1\linewidth,innerleftmargin=3pt,innerrightmargin=3pt, userdefinedwidth=1\linewidth, linewidth=0pt, linecolor=white, usetwoside=false]}
        {\end{mdframed}}

        \renewenvironment{Verbatim}[1][\unskip]
        {\begin{alltt}\smaller}
        {\end{alltt}}
    

    % Help prevent overflowing lines due to urls and other hard-to-break 
    % entities.  This doesn't catch everything...
    \sloppy

    % Document level variables
    \title{weekly\_update\_02\_feb\_16}
    \date{February 17, 2014}
    \release{}
    \author{Christian Carmona}
    \renewcommand{\releasename}{}

    % TODO: Add option for the user to specify a logo for his/her export.
    \newcommand{\sphinxlogo}{}

    % Make the index page of the document.
    \makeindex

    % Import sphinx document type specifics.
     


% Body

    % Start of the document
    \begin{document}

        
            \maketitle
        

        


        
        \part{Weekly update 02: feb 10 - feb 16, 2014}I dedicated this week to analyze different models for the term structure
of interest rates, identifying the market and academic standards for
quoting interest rates, and coding two functions: 1) for obtaining spot
(zero-coupon) rates from yield to maturity, and 2) for pricing bonds
given a of zero-coupon yield curve.

\textbf{Note on coding language:} I decided to use R as the main coding
language. The reason for this is that my research is widely applicable
to the functions carried by the International Reserve Division in the
Central Bank of Mexico, my current employer, and they use R as the main
software for data analysis. However, If it is necessary to create
equivalent functions in Python to cover course objectives, I am open to
translate all my developed code.

The main inputs of portfolio selection models are the expected values
and covariances of the assets under consideration. In the case of
fixed-income portfolios, the prices of the instruments are function of
the interest rates observed in the market. Therefore, a theoretical
model for the evolution of bond prices over time is needed.\section{Term Structure of interest rate}The \emph{term structure of interest rates} at a given time is the
functional relationship between spot interest rates and term to maturity
{[}Wilhelm, 1995{]}.

For the term structure to be observable in the market, zero-coupon bonds
of different maturities must trade. In reality, only zero-coupon bonds
of very few maturities (typically smaller than a year) trade. But an
abundance of traded coupon bonds exist in the fixed income markets.
Coupon bonds can be thought of as portfolios of zero-coupon bonds.
Hence, only specific portfolios of zero-coupon bonds trade.

The problem is then to extract individual zero-coupon prices from the
prices of particular portfolios of zero-coupon bonds. From these
(theoretical) zero-coupon bond prices the corresponding spot rates can
then be calculated.

Common alternatives for extracting Yield Curves from Bond Prices
{[}Munk, 2011{]}:

Bootstrapping

Cubic Spline

The Nelson-Siegel parametrization

\textbf{Market quoting for bond prices}

Consider that we can observe two types of bonds in the market: 1)
zero-coupon bonds with maturity of at most ne year, 2) Coupon with
semi-annual coupons payments with maturities ranging from 0 to 30 years.

Denote $z_t$: t-year continuously compounding interest rate. The price
at time t, for a bond maturing at time T with a face value of \$1 is:

For zero-coupon bonds:

\[ B_t = e^{-t * z_t}\]

For coupon bonds:

\[ P_t = c \sum\limits_{t_i} e^{-t_i * z_{t_i}} + e^{-t * z_t} = c \sum\limits_{t_i} B_{t_i} + B_{t} \]

where: $c$: coupon-rate, and $t_i$: time of coupon payments;
$0 < t_i \leq t$

However, the market standard for quoting prices of these instruments are
not their prices, neither the zero-coupon rates.

In the following, we examine the case of US governments bonds. In
practive, we will find the following:

For zero-coupon bonds (T-bills), the quotes yield is the following
``adjusted discount rate'':

$d_t = \frac{1-B_t}{B_t}*\frac{365}{t}$

For coupon bonds (T-notes), the quote is the yield to maturity of the
bond, i.e.~the rate $y_t$ such that the current price of the bond is
equal to present value of the cashflows discounted with that same rate:

$y_t$ is such that:
$P_t = c \sum\limits_{t_i} e^{-t_i * y_t} + e^{-t * y_t}$

If we have the quotes for $d_t$ and $y_t$, we can easily obtain the
prices of the bonds: $B_t$ and $P_t$, and then calculate the spot rates
using one of the three aforementioned methods.

\begin{center}\rule{3in}{0.4pt}\end{center}

\textbf{Bootstrapping}

This method is an iterative procedure for getting spot rates from given
prices of coupon bonds.

Suppose we have the following prices: $B_{0.5}$, $B_{1}$, $P_{1.5}$,
$P_{2}$, $P_{2.5}$, $P_{3}$, \ldots{}.

It is easy to get spot rates from zero-coupon bonds by:

$z_t = -t * log(B_t)$

for $t=0.5$ and $t=1$, we can directly obtain $z_{0.5}$ and $z_{1}$.

For $t=1.5$, we can use $P_{1.5}$ and get $B_{1.5}$ from the expression:
\[ P_{1.5} = c * (B_{0.5}+B_{1}+B_{1.5}) + B_{1.5}\] then,
\[ B_{1.5} = \frac{1}{1+c} * ( P_{1.5} - c * (B_{0.5}+B_{1})) \] which
is a zero-coupon bond, from which we can get $z_{1.5}$

this algorithm is repeated for t=\{ 1.5, 2, 2.5, 3, \ldots{} \}

\begin{center}\rule{3in}{0.4pt}\end{center}

\textbf{Cubic Spline} {[}McCulloch, 1975{]}

This method adjust a fuction $\bar{B}(t)$ to obtain the values of the
discount factors for all times $t$,

Divide the time domain into intervals defined by the ``knot points''
$0 = \tau_{0} \leq \tau_{1} \leq ... \leq \tau_{k}$

The spline approximation of the discount function is given by:
\[\bar{B}(t)=\sum\limits_{j=0}^{k-1} G_j(t) I_j(t)\]

where the $G_j(t)$'s are basis functions and the $I_j(t)$'s are step
functions:
\[I_j(t) = \begin{cases} 0 & \mbox{if } t \geq \tau_j \\ 1 & \mbox{otherwise } \end{cases}\]

Hence:
\[\bar{B}(t) = \begin{cases} G_{0}(t) & \mbox{for } t \in[\tau_0,\tau_1) \\ G_{0}(t)+G_{1}(t) & \mbox{for } t \in[\tau_1,\tau_2) \\ ... & ... \\ G_{0}(t)+G_{1}(t)+...+G_{k-1}(t) & \mbox{for } t \geq \tau_{k-1} \\ \end{cases}\]

We demand that the $G_j(t)$'s are continuous, differentiable and ensure
a smooth transition in the knot points.

This approach use a cubic polinomial spline:
\[ G_j(t) = \alpha_j + \beta_j (t-\tau_j) + \gamma_j (t-\tau_j)^2 + \delta_j (t-\tau_j)^3 \]
where $\alpha_j, \beta_j, \gamma_j, \delta_j$ are constants.

The smooth transition demand continuity in the function and its two
derivatives: \[\bar{B}(\tau_1-)=\bar{B}(\tau_1+)\]
\[\bar{B}'(\tau_1-)=\bar{B}'(\tau_1+)\]
\[\bar{B}''(\tau_1-)=\bar{B}''(\tau_1+)\]

with these assumptions, we can derive some of the constants, reducing
the expression to:
\[\bar{B}(t)=1+\beta_0 t + \gamma_0 t^2 + \delta_0 t^3 + \sum\limits_{j=1}^{k-1} \delta_j (t-\tau_j)^3 I_j(t) \]

To obtain the remaining constants, we adjust a linear model using the
observed prices of bonds $B_t$ and $P_t$
\[ P_t = c \sum\limits_{0< t_i \leq t } \bar{B}(t_i) + \bar{B}(t) + \epsilon_t \]

\begin{center}\rule{3in}{0.4pt}\end{center}

\textbf{The Nelson-Siegel parametrization} {[}Nelson and Siegel, 1987{]}

This approach is based on a parametrization on the structure of the
forward rates:
\[ \bar{f}(t)=\beta_0+\beta_1 e^{-t/\theta}+\beta_2\frac{t}{\theta}e^{-t/\theta} \]
where $\beta_0,\beta_1,\beta_2,\theta$ are constants to be estimated and
apply for all maturities.

Then, the term structure of zero-coupon rates is given by:
\[ \bar{z}_t = \frac{1}{t} \int\limits_{0}^{t} \bar{f}(u) = \beta_0+\beta_1 \frac{ 1- e^{-t/\theta} }{t/\theta} + \beta_2 (\frac{1-e^{-t/\theta}}{t/\theta}-e^{-t/\theta}) \]

which we can rewrite as:
\[ \bar{z}_t = a + b \frac{ 1- e^{-t/\theta} }{t/\theta} + c e^{-t/\theta} \]

the parameters $a,b,c$ are obtained using the observed prices $B_t$ for
short terms using fitting a linear model:

\[ z_{t_i} = a + b \frac{ 1- e^{-t_i/\theta} }{t_i/\theta} + c e^{-t_i/\theta} + \epsilon_i\]

and for the coupon bonds, the expression is slightly more complicated
and requires the use of Generalized Linear Models, because it is not
linear in the unknown parameters.\section{Model Selection for my project}I decided to adopt \textbf{Bootstrapping method} for modeling the term
structure in my project in view of the following:

\begin{enumerate}
\def\labelenumi{\arabic{enumi}.}
\itemsep1pt\parskip0pt\parsep0pt
\item
  Bootstrapping is a widely used methodology among practitioners
\item
  The underlying assumptions to get zero-coupon rates are consistent
  with the non-arbitrage criteria
\item
  For cubic spline, the value of the discount function for some
  maturities can often be determined by pure no-arbitrage arguments as
  utilized in the bootstrapping approach. The discount function
  estimated with cubic splines will not necessarily match those values
  so applications of the estimated function will not respect the
  fundamental no-arbitrage pricing principle.
\item
  For cubic spline, there is no guarantee whatsoever that the discount
  function estimated using cubic splines has an economically credible
  form. In particular, the discount function should be positive and
  decreasing (which will ensure positive forward rates), but there is
  nothing in the approach ensuring that.
\item
  For cubic and Nelson-Siegel, small variations in the input bond prices
  may have a substantial effect on the estimated discount function and
  yield curve. In particular, a change in the input price of a
  short-maturity bond may even affect the long-maturity end of the
  estimated curves.
\item
  For Nelson-Siegel, the necessity of incorporate estimation by
  Generalized Linear Models is way more complicated and computationally
  expensive.
\item
  For Nelson-Siegel, the slope and the curvature factors decay rapidly
  to zero as the maturity increases. Hence, it is difficult to fit
  medium- and long-term yields. An additional curvature term can be
  added to reduce this problem, but the complexity in the model is
  increased even more.
\end{enumerate}\section{Coding}I developed the following two fuctions in R for implementing
Bootstrapping method using historical real data\begin{verbatim}
##################################################

interpolation <- function( x, fx, x_new, method=c("linear","exp","log","spline")[1] ) {
  # This function obtain the interpolated values in x_new
  # For a given a set of points (x,fx)
  
  
  x_new_bucket <- rep(1,length(x_new))
  for( i in seq(x) ) {
    x_new_bucket <- ifelse(x_new>x[i], i+1, x_new_bucket)
  }
  x_new_bucket <- ifelse(x_new==x[1], 2, x_new_bucket)
  temp <- !is.element(x_new_bucket,c(1,length(x)+1))
  fx_new <- as.numeric(NULL)
  if(method=="linear") {
    # Smoothing via "Linear interpolation"
    w <- rep(NA,length(x_new))
    w[temp] <- (x[x_new_bucket[temp]]-x_new[temp]) / (x[x_new_bucket[temp]] - x[x_new_bucket[temp]-1])
    fx_new <- rep(NA,length(x_new))
    fx_new[temp] <- fx[x_new_bucket[temp]-1] * w[temp] + fx[x_new_bucket[temp]] * (1-w[temp])
  }
  if(method=="exp") {
    # Smoothing via "exponential interpolation"
    w <- rep(NA,length(x_new))
    w[temp] <- (x[x_new_bucket[temp]]-x_new[temp]) / (x[x_new_bucket[temp]] - x[x_new_bucket[temp]-1])
    fx_new[temp] <- log( exp(fx[x_new_bucket[temp]-1]) * w[temp] + exp(fx[x_new_bucket[temp]]) * (1-w[temp]) )
  }
  if(method=="log") {
    # Smoothing via "Logarithmic interpolation"
    w <- rep(NA,length(x_new))
    w[temp] <- (x[x_new_bucket[temp]]-x_new[temp]) / (x[x_new_bucket[temp]] - x[x_new_bucket[temp]-1])
    #fx_new[temp] <- exp( log(fx[x_new_bucket[temp]-1]) * w[temp] + log(fx[x_new_bucket[temp]]) * (1-w[temp]) )
    fx_new[temp] <- fx[x_new_bucket[temp]-1]^w[temp] * fx[x_new_bucket[temp]]^(1-w[temp])
  }
  if(method=="spline") {
    # Smoothing via "Smoothing spline"
    smooth_model <- smooth.spline(cbind(x,fx))
    fx_new <- predict(smooth_model,x_new)$y
  }
  return(fx_new)
}

##################################################

##################################################

zero_from_yield_bootstrap <- function ( nodes, ytm_curve, smooth=c("linear","exp","log","spline")[2] ) {
  # Function to calculate zero-coupon rates, z_t, using bootstrapping
  # using data acording to market quotes, y_t
  #   for t > 1 year: anualized yield to maturity for bonds paying semiannual coupons
  #   fot t <=1 year: adjusted discount rate, i.e. ((100-P)/P)*(365/days to maturity)
  # the output rate will be annual and continuosly compounded
  
  # Input:
  #   ytm_curve: a numeric vector with yields in the market standard
  #   nodes: a numeric  vector with the terms of ytm_curve in years
  # Output
  #   zero_curve_boot :a numeric vector with the corresponding zero-coupon yields
  
  ytm_curve <- as.numeric(ytm_curve)
  
  nodes_t <- as.numeric(substr( nodes , 1, nchar(nodes)-1 ))
  nodes_t[substr( nodes , nchar(nodes), nchar(nodes) )=="m"] <- nodes_t[substr( nodes , nchar(nodes), nchar(nodes) )=="m"]/12
  if(!all(nodes_t==sort(nodes_t,decreasing=F))) {
    stop("Introduced values have to be increasing with respect to 'nodes'")
  }
  
  # Get rid of NA's
  nodes_orig_t <- nodes_t
  nodes <- nodes[!is.na(ytm_curve)]
  nodes_t <- nodes_t[!is.na(ytm_curve)]
  ytm_curve <- ytm_curve[!is.na(ytm_curve)]
  
  # Validates that the vector has first, last and one additional node.
  if(
    !( all( is.element( nodes_orig_t[c(1,length(nodes_orig_t))] , nodes_t[c(1,length(nodes_t))] ) ) & length(nodes)>3 )
  ) { return( rep(NA,length(nodes_orig_t)) ) }
  
  # Interpolation of values in the input yields for terms that are not specified but needed in the algorithm
  ytm_pred_t <- sort(unique( c( nodes_t, seq(0.5,max(nodes_t),0.5) ) ))
  ytm_pred <- interpolation(x=nodes_t,fx=ytm_curve,x_new=ytm_pred_t,method=smooth)

  # Validates that the interpolations does not give negative values
  if(any(ytm_pred<0)) { stop("The interpolation method is calculating negatives yields") }
  
  # Zero-coupon yield calculation via bootstrappping
  # Assumptions:
  # 1) Quotation according to market stndards:
  #   1.1) for t<=1 instruments are zero-coupon bonds
  zero_curve_boot <- ytm_pred
  zero_curve_boot[ytm_pred_t<=1] <- ( 1 + ytm_pred[ytm_pred_t<=1] * ytm_pred_t[ytm_pred_t<=1] ) ^ (1/ytm_pred_t[ytm_pred_t<=1])-1
  zero_curve_boot[ytm_pred_t>1] <- NA
  #   validation
  if( any(round( ( (1+ytm_pred[ytm_pred_t<=1]*ytm_pred_t[ytm_pred_t<=1])^(-1) ) - ( (1+zero_curve_boot[ytm_pred_t<=1])^(-ytm_pred_t[ytm_pred_t<=1]) ) , 12 ) >0) ) {
    stop("Zero-coupon rates were not calculated correctly for terms t<=1")
  }
  
  #   1.2) for t>1 instruments are coupon bonds, with payments at: 0.5, 1, 1.5, 2, 2.5,...
  # ytm are nominales semi-anually compounded
  node_coupon <- is.element(ytm_pred_t,seq(0.5,max(ytm_pred_t),0.5))
  
  for( node_i in ytm_pred_t[ytm_pred_t>1] ) {
    
    ytm_i <- ytm_pred[match(node_i,ytm_pred_t)]
    
    # cupon rate that makes bond price=100
    cpn_i <- ytm_i/2
    
    # zero-rate for node_i
    zero_curve_boot[match(node_i,ytm_pred_t)] <-    
      ( ( 1-cpn_i*sum( ( 1 + zero_curve_boot[node_coupon & (ytm_pred_t<node_i)] )^(-ytm_pred_t[node_coupon & (ytm_pred_t<node_i)]) ) ) / ( 1+cpn_i ) )^(-1/node_i) - 1
    
    # Validation
    if( 100 != round( sum(100*cpn_i*(1+zero_curve_boot[node_coupon & (ytm_pred_t<=node_i)])^( -ytm_pred_t[node_coupon & (ytm_pred_t<=node_i)] ) ) + 100*(1+zero_curve_boot[ytm_pred_t==node_i])^-ytm_pred_t[ytm_pred_t==node_i] ,12) ) {
      stop("Zero-coupon rates were not calculated correctly for terms t>1")
    }
    #plot(zero_curve_boot,pch=20,col=2)
    #points(ytm_pred,pch=20,col=4)
    #legend("top",c("zero","ytm"),col=c(2,4),pch=19)
    
  }
  
  return( zero_curve_boot[match(nodes_orig_t,ytm_pred_t)] )
}

##################################################
\end{verbatim}

    % Make sure that atleast 4 lines are below the HR
    \needspace{4\baselineskip}

    
        \vspace{6pt}
        \makebox[0.1\linewidth]{\smaller\hfill\tt\color{nbframe-in-prompt}In\hspace{4pt}{[}{]}:\hspace{4pt}}\\*
        \vspace{-2.65\baselineskip}
        \begin{ColorVerbatim}
            \vspace{-0.7\baselineskip}
            \begin{Verbatim}[commandchars=\\\{\}]

\end{Verbatim}

            
                \vspace{0.3\baselineskip}
            
        \end{ColorVerbatim}
    

        

        \renewcommand{\indexname}{Index}
        \printindex

    % End of document
    \end{document}


